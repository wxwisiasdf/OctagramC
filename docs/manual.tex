\documentclass[12pt, letterpaper]{article}
\usepackage{listings}
\usepackage{color}
\lstset{frame=tb,
    language=C,
    aboveskip=3mm,
    belowskip=3mm,
    showstringspaces=false,
    columns=flexible,
    basicstyle={\small\ttfamily},
    numbers=none,
    numberstyle=\tiny\color{cyan},
    keywordstyle=\color{blue},
    keywordstyle=[2]\color{red},
    commentstyle=\color{green},
    stringstyle=\color{red},
    breaklines=true,
    breakatwhitespace=true,
    tabsize=3
}
\lstdefinelanguage{SSA}{
    sensitive,
    moredelim=[s][{\itshape\color[rgb]{0,0,0.75}}]{\#[}{]},
    morestring=[b]{"},
    alsodigit={},
    alsoother={},
    alsoletter={\$\_},
    %types
    morekeywords=[1]{u0,u4,u8,u16,u32,u64,u128},
    morekeywords=[1]{i0,i4,i8,i16,i32,i64,i128},
    morekeywords=[1]{f0,f4,f8,f16,f32,f64,f128},
    morekeywords=[1]{fn},
    %params
    morekeywords=[1]{none},
    %instructions
    morekeywords=[2]{load_from,store_from,drop,add,sub,mul,div,mod,rem,alloca,and,xor,not,or,lshift,rshift,call,jump,ret,branch,eq,neq,gt,gte,lt,lte}
    %indicators
    morekeywords=[2]{size,align}
}
\usepackage{tabularx}
\title{OctagramC manual}
\author{Leaf}
\date{May 2023}
\begin{document}
\maketitle
\newpage
%
TODO: binop use only tmpid for lhs
%
\newpage
\section{SSA System}
%
The SSA intermediate representation is composed of tokens, which represent an
abstract instruction to perform.
%
\newpage
\begin{center}
\begin{table}
\begin{tabularx}{\columnwidth}{|X|X|}
    \hline
    none & No parameter is specified/required. May be often seen along a call that might have side effects \\
    \hline
    variable & A variable is a constant representing a named address, that is, an address that has a name \\
    \hline
    constant & Represents a constant value that doesn't change \\
    \hline
    tmpvar & A temporal holding variable, may only be assigned once \\
    \hline
    string\_literal & A form of a constant wich is anonymously named - that is, it's name is generated on the fly by the compiler. And holds a string of data \\
    \hline
\end{tabularx}
\label{table: SSA Variable Parameters}
\end{table}
\end{center}
%
\begin{center}
\begin{table}
\begin{tabularx}{\columnwidth}{|X|X|X|}
    \hline
    load\_from & \textbf{value} = load\_from \textbf{address} & Loads \textbf{value} with the one in \textbf{address} \\ 
    \hline
    store\_from & \textbf{address} = store\_from \textbf{value} & Stores \textbf{value} into \textbf{address} \\ 
    \hline
\end{tabularx}
\label{table: SSA Instructions}
\end{table}
\end{center}
%
\newpage
\subsection{Example with cc\_zalloc}
The compiler transforms the following:
\begin{lstlisting}[language=C]
void* cc_zalloc(size_t size)
{
    void* p = calloc(size, 1);
    if (p == 
            ((void *)0)
                )
        do { cc_abort_1("util.c", 125); abort(); } while (0);
    return p;
}
\end{lstlisting}
Into an SSA form:
\begin{lstlisting}[language=SSA]
fn cc_zalloc {
    u32 var_p = alloca size u32 0, align u0 none
    u32 tmp_7 = call u32 var_calloc(u32 var_size, u32 1, )
    u32 var_p = store_from u32 tmp_7
    drop tmp_7
    u32 tmp_12 = u32 var_p eq u32 0
    drop tmp_12
L_256800:
    u0 none = call u32 var_cc_abort_1(u32 string_19_"util.c", u32 125, )
    u0 none = call u32 var_abort()
    jump u32 L_60192
    ret u32 var_p
}
\end{lstlisting}
%
\end{document}
